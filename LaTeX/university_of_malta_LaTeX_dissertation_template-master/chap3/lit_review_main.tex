% Notes:

\chapter{Literature Review}

\section{Classification in Astronomy}

Applying the concepts of Machine Learning in the field of Astronomy is by no means a new endeavour.
Ever since the advent of telescopes and surveys generating valuable data, research which exploits this data quickly followed suit.
Promising attempts to classify galaxies using machine learning date back to 1992 \citep{storrie-lombardi_morphological_1992}.
Other applications include quasar \citep{richards_bayesian_2015} and multi class \citep{wang_j-plus_2022}\citep{baqui_minijpas_2021} classifications.

\section{Establishing classes}
% Mention classes being used 
% mention barchi paper? results leave room for improvement.


\section{Data Augmentation Techniques}
    % - rotations,flips \citep{khramtsov_deep_2019} 
    % - smote? \citep{barchi_machine_2020} 
    % - oversampling better for deep learning?
    % - example: \citep{reza_galaxy_2021}

\section{Pre-processing to improve performance}
    - cropping, rescaling (no noticeable effect?)\citep{dieleman_rotation-invariant_2015}
